
\documentclass{article}
\usepackage{bm,amsmath}
\usepackage{enumerate}
\usepackage{gensymb}
\usepackage{titling}
\usepackage{enumitem}
\usepackage{tfrupee}
\usepackage{physics}
\newcommand{\myvec}[1]{\ensuremath{\begin{pmatrix}#1\end{pmatrix}}}
\newcommand{\mydet}[1]{\ensuremath{\begin{vmatrix}#1\end{vmatrix}}}
\providecommand{\sbrak}[1]{\ensuremath{{}\left[#1\right]}}
\providecommand{\brak}[1]{\ensuremath{\left(#1\right)}}
\providecommand{\cbrak}[1]{\ensuremath{\left\{#1\right\}}}
\begin{document}
\title{\textbf{CBSE QUESTIONS}}
\date{}
\maketitle{}
\section{Linear Programming}
\begin{enumerate}
\item A factory manufactures two types of screws A and B, each type requiring the use of two machines, an automatic and a hand-operated. It takes $4$ minutes on the automatic and $6$ minutes on the hand-operated machines to manufacture a packet of screws 'A' while it takes $6$ minutes on the automatic and $3$ minutes on the hand-operated machine to manufacture a packet of screws 'B'. Each machine is available for at most $4$ hours on any day. The manufacturer can sell a packet of screws $‘A’$ at a profit of $70$ paise and screws 'B' at a profit of $< 1$. Assuming that he can sell all the screws he manufactures, how many packets of each type should the factory owner produce in a day in order to maximize his profit ? Formulate the above LPP and solve it graphically and find the maximum profit ?
\end{enumerate}
\end{document}
